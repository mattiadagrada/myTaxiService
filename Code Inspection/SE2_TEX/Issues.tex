\section{Code Inspection}
\subsection{getLaunchInfo}
	\subsubsection{Description}
	\begin{description}
	\item[Location]: appserver/admingui/common/src/main/java/org/glassfish/admingui/common/handlers/ApplicationHandlers.java
	\item[Name]: getLaunchInfo(String appName, String contextRoot, Map oneRow)
	\item[Modifiers]: private static void
	\item[Start Line]: 368
	\end{description}
	\subsubsection{Issues}
	\begin{description}
		\item[11] One statement statements are not surrounded by curly braces
			\begin{itemize}
			\item If at line 378
			\item If at line 396
			\end{itemize}
		\item[13] Multiple lines exceed the 80 characters (spaces included).
			This is not a major issue, because none of them	exceeds the limit of 120 characters on the same line and
			the readability is not compromised.
			\begin{itemize}
				\item Line 369
				\item Line 380
				\item Line 391
				\item Line 394
				\item Line 395
			\end{itemize}
		\item[18] This is a private method, so it is comprehensible that no javadoc has been provided, but
		without comments it is quite hard to understand what the method does.
		\item[52] There are no exceptions caught or thrown explicitly. There are extra controls to avoid method
		calls on null objects and some other measures to avoid execution errors, but they might not be enough.
	\end{description}
\subsection{if(appPropsMap != null)}
	\subsubsection{Description}
	\begin{description}
	\item[Location]: appserver/admingui/common/src/main/java/org/glassfish/admingui/common/handlers/ApplicationHandlers.java
	\item[Name]: if(appPropsMap != null)
	\item[Start Line]: 617
	\end{description}
	\subsubsection{Issues}
	\begin{description}
		\item[9] A tab is used to indent line 617
		\item[11] One statement statements are not surrounded by curly braces
		\begin{itemize}
			\item For at line 636
			\item If at line 639
		\end{itemize}
		\item[14] Two lines exceed the 120 characters (spaces included).
		\begin{itemize}
			\item Line 629 (151 chars)
			\item Line 631 (146 chars)
		\end{itemize}
		\item[17] There is a mistake in the indentation: the wrapping If is not aligned with the upper statement.
		\item[18] In this piece of code (that is a particular part of a method) there are no sensible comments.
		Unfortunately also the javadoc is incomplete and is not clear the role of this public method.
		\item[23] The javadoc for this method is incomplete, practically it is totally missing.
		\item[33] In the For at line 618 there are several declarations and initializations mixed with "put"
		operations in the output row.
		\item[53] There is a single try-catch block for the whole piece of code. It is not very elegant, but
		it avoids errors. The problem is that the general exception is catched, but no actions are properly taken.
	\end{description}