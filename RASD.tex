\documentclass{article}
\begin{document}
\section{Overall Description}
	\subsection{Product perspective}
		We want to release both a web and a mobile application, avoiding deep
		integrations with other existing systems in order to keep an high level
		of compatibility, especially the mobile application, that we want to
		release for all major mobile operative systems. The applications will not
		have any internal interface for administration but they will be only user based.
		The application will provide APIs for a faster, safer and better integrated 
		implementation of new features, based on the basic functionalities
		offered by the system.
	\subsection{User characteristics}
		We expect to have two different types of user with distinct experiences. 
		\begin{description}
			\item[Passenger] who needs to call a taxi for a ride (that is occasional, threated
			as a guest, or habitual, who can register himself to speed up request submissions).
			This user must have access to the Internet and be able to use
			a web browser or have installed our mobile application on his smartphone;
			\item[Taxi Driver] who wants to offer its service through our system.
			This user must have access to Internet and have	installed our mobile
			application on his smartphone.
		\end{description}
	\subsection{Constraints}
		\subsubsection{Regulatory policies}
			Personal information (locations included) obtained from the	user will be stored
			in our databases if strictly necessary, to provide the best possible experience
			with our service. No information will be disclosed to any other company or
			used for any other purpose.
		\subsubsection{Hardware limitations}
			MyTaxiService's web application will be available on every device with an Internet
			connection and a browser installed. The mobile application (necessary for taxi
			drivers) will have the following requirements:
			\begin{itemize}
				\item 256MB total RAM
				\item 50MB of free space on Disk
				\item Internet Connection
				\item Operative System:
					\begin{itemize}
						\item Android 2.3 ''Gingerbread'' or later
						\item Windows Mobile 7.0 or later
						\item iOS 6.0 or later
					\end{itemize} 
			\end{itemize}
			It is also recommended to activate the GPS for a optimal localization.
		\subsubsection{Interfaces to other applications}
			MyTaxiService will provide an API library to allow external developers to
			integrate their applications with our system.
		\subsubsection{Parallel operation}
			The system must support parallel operations from different users. Information
			integrity is fundamental to provide a first-rate service, so the mechanism of
			supply and demand will have to be highly affordable.
		\subsubsection{Documents related}
			\begin{itemize}
				\item RASD (Requirements and Analysis Specification Document)
				\item DD (Design Document)
				\item User's Manual
				\item Testing report
			\end{itemize}
	\subsection{Assumptions and Dependencies}
		\begin{itemize}
			\item There is not an administrator or a privileged user. We think that is not
			necessary a hierarchy of users to keep the system safe;
			\item There is not any dependence between users;
			\item Every registered user can request a single taxi;
			\item Guest user is identified through his public IP address;
			\item Guest users will be asked to provide basic personal informations during the
			request's submission
			\item The position of the user can always be determined, via browser or GPS
			(see HTML5 Geolocation), and has a good approximation. Otherwise the service will not
			be accessible;
			\item A taxi driver can access the service only if he is signed up. During
			registration he will have to provide, in addition to the usual personal information,
			the identification number of his taxi;
			\item Through the ID number of a taxi is possible to retrieve some informations (like
			its location);
			\item A taxi driver can give and remove availability in every moment;
			\item If a taxi driver does not accept a call within 1 minute after the notification,
			a rejection will be automatically recorded in the database and the request will be
			forwarded to another taxi;
			\item After 3 rejections (voluntary or involuntary) the taxi availability will be 
			revoked automatically;
			\item When a user (registered or guest) requests a taxi, his position is taken automatically.
			If he does not agree, his request will not be forwarded.
			\item After a request, a user (registered or guest) can not make other requests in the next 30 minutes;
			\item A user must be signed up to make a reservation
			\item Reservations must be taken at least 2 hours in advance from the time of taxi's request,
			otherwise the reservation will not be accepted
			\item A user that registered a reservation for a taxi can not make requests within 30 minutes
			before the reservation
		\end{itemize}
	\subsection{Future possible implementation}
		The taxi sharing option: this means that the user is ready to 
		share a taxi with others if possible, thus sharing the cost of the ride. In this case 
		the user is required to specify the destination of all rides which he/she wants to 
		share with others. If others are willing to start a shared ride from the same zone 
		going in the same direction, then the system arranges the route for the taxi 
		driver, defines the fee for all persons sharing the taxi and informs the passengers 
		and the taxi driver.
\end{document}
Dubbi su Regulatory e Hardware limitations