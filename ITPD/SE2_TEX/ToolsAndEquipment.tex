\section{Tools and Test Equipment Required}
	\subsection{Arquillian}
	\textit{'The mission of the Arquillian project is to provide a simple test harness that developers can use
		to produce a broad range of integration tests for their Java applications...'}
	\begin{figure}[h!]
		\begin{center}
			\href{http://arquillian.org/}{\includegraphics[scale=1]{../SE2_IMAGES/arquillian}}
		\end{center}
	\end{figure}
	Arquillian, a testing framework developed at JBoss.org, empowers the developer to write integration
	tests for business objects that are executed inside a container or that interact with the container as a client.
	The container may be an embedded or remote Servlet container, Java EE application server, Java SE CDI
	environment or any other container implementation provided. Arquillian strives to make integration
	testing no more complicated than basic unit testing.
	\\
	\\
	It's obvious that this solution represents a good choice mainly if JEE is the environment chosen to develop
	our project, as we suggested in Design Document, otherwise it could be not compatible with the code written.
	\subsection{Manual testing}
	The manual testing of the interactions is always available. It may be not a wise choice, because there are
	no aids nor automations, but it could be the only way to test some parts of your code. The manual testing
	could be also slow and unsafe (paradox: it could be bugged itself), so we do not recommend to use it as
	a universal choice, but it could be a great solution if used in combination with a software tool to cover
	what the tool can not manage easily.