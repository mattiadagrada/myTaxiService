\section{Integration Strategy}
	\subsection{Entry Criteria}
	To proceed at the Integration Testing level treated in this document is necessary that every
	component of the system has been previously unit tested with a positive result.
	We must be sure that all the methods inside the components, here considered like black boxes,
	work as expected.
	This is a necessary constraint that must be satisfied, because the purpose of the tests discussed in this
	document is to check the interaction between components (that's why it is called "integration testing").
	\subsection{Elements to be Integrated}
	\textit{Premise}: for a more detailed description of the system structure, refer to myTaxiService Design Document,
	sections 2.3 and 2.7.
	\\
	\\
	Our system has been originally divided in four subsystems:
	\begin{itemize}
		\item Client
		\item Frontend System
		\item Backend System
		\item Database
	\end{itemize}
	Every subsystem is composed by a variable number of components and every single component has been
	unit tested individually. In this document we want to test the interaction between different components,
	so we are going to test all the interfaces provided by the components, dedicating a test case to every
	different interaction that involves an interface. Practically we want to test all the couples of components
	that interact between them.
	\newpage
	\begin{figure}[h!]
		\begin{center}
			\includegraphics[width=\textwidth]{../SE2_IMAGES/system}
			\caption{A particular component view}
		\end{center}
	\end{figure}
	\subsection{Integration Testing Strategy}
	We have chosen a bottom-up strategy. We think it's the best choice because represents a more gradual
	approach to the system testing. It also allows to find an eventual issue at the deepest level of
	integration, giving the possibility to intervene immediately at the correct level, without having to
	resort to a tougher analysis.
	\subsection{Sequence of Component/Function Integration}
		\subsubsection{Software Integration Sequence}
		The order of the tests into the tables (and the growing ID number too) represents also the sequence in
		which the components will be integrated within each subsystem.
		\paragraph{Integration Test of the Database - software}
		\begin{tabular}{p{2cm} | p{10cm}} \hline
			\textbf{ID} & \textbf{Integration Test} \\ \hline
			I1 & Request $\rightarrow$ Database \\ \hline
			I2 & Reservation $\rightarrow$ Database \\ \hline
			I3 & Guest $\rightarrow$ Database \\ \hline
			I4 & RegisteredPassenger $\rightarrow$ Database \\ \hline
			I5 & TaxiDriver $\rightarrow$ Database \\ \hline
		\end{tabular}
		\paragraph{Integration Test of the Backend system - software}
		\begin{tabular}{p{2cm} | p{10cm}} \hline
			\textbf{ID} & \textbf{Integration Test} \\ \hline
			I6 & Request $\rightarrow$ Area \\ \hline
			I7 & Guest $\rightarrow$ Request \\ \hline
			I8 & RegisteredPassenger $\rightarrow$ Request \\ \hline
			I9 & RegisteredPassenger $\rightarrow$ Reservation \\ \hline
			I10 & TaxiDriver $\rightarrow$ Request \\ \hline
			I11 & TaxiDriver $\rightarrow$ QueueManager \\ \hline
		\end{tabular}
		\paragraph{Integration Test of the Frontend system - software}
		\begin{tabular}{p{2cm} | p{10cm}} \hline
			\textbf{ID} & \textbf{Integration Test} \\ \hline
			I12 & Client $\rightarrow$ Guest \\ \hline
			I13 & Client $\rightarrow$ RegisteredPassenger \\ \hline
			I14 & MobileClient $\rightarrow$ TaxiDriver \\ \hline
		\end{tabular}
		\paragraph{Integration Test of the UserClient - software}
		\begin{tabular}{p{2cm} | p{10cm}} \hline
			\textbf{ID} & \textbf{Integration Test} \\ \hline
			I15 & MobileClient $\leftrightarrow$ System \\ \hline
			I16 & WebClient $\leftrightarrow$ System \\ \hline
		\end{tabular}
		\subsubsection{Subsystem Integration Sequence}
			When all the subsystems have been individually tested, the can be integrated one by one
			attaching, the subsystems sequentially in the following order:
			\\ \\
			\begin{tabular}{p{2cm} | p{10cm}} \hline
				\textbf{Order} & \textbf{Subsystem} \\ \hline
				1 & Database \\ \hline
				2 & BackendSystem \\ \hline
				3 & FrontendSystem \\ \hline
				4 & Client \\ \hline
			\end{tabular}
			\\ \\
			When the integration testing reaches the top level and the whole system has been integrated,
			some integration test procedures can be individuated analysing the main functionalities that
			our system has to provide.
			\\ \\
			\begin{tabular}{p{4cm} | p{8cm}} \hline
				\textbf{Test Procedure ID} & TP1 \\ \hline
				\textbf{Purpose} & This test procedure verifies whether the registration process works:
				\begin{itemize}
					\item can access both the Clients
					\item can fill the registration form
					\item can store a RegisteredPassenger
				\end{itemize}
				\\ \hline
				\textbf{Procedure Steps} & I4T1, I12T1, I15T1 and I16T1(MobileClient), I17T1 and I18T1(WebClient) \\ \hline
			\end{tabular}
			\vspace{1cm}
			\begin{tabular}{p{4cm} | p{8cm}} \hline
				\textbf{Test Procedure ID} & TP2 \\ \hline
				\textbf{Purpose} & This test procedure verifies whether the login procedure works
					for the RegisteredPassenger:
					\begin{itemize}
						\item can access both the Clients
						\item can fill the login form
						\item can retrieve the RegisteredPassenger
					\end{itemize}
				\\ \hline
				\textbf{Procedure Steps} & I4T2, I12T2, I15T1 and I16T1(MobileClient), I17T1 and I18T1(WebClient)\\ \hline
			\end{tabular}
			\vspace{1cm}
			\begin{tabular}{p{4cm} | p{8cm}} \hline
				\textbf{Test Procedure ID} & TP3 \\ \hline
				\textbf{Purpose} & This test procedure verifies whether the login procedure works
				for the TaxiDriver and he is able to manage his availability:
				\begin{itemize}
					\item can access the MobileClient
					\item can fill the login form
					\item can retrieve the TaxiDriver
					\item can give his availability
					\item can revoke his availability
				\end{itemize}
				\\ \hline
				\textbf{Procedure Steps} & I5T2, I14T1, I11T2, I14T4, I11T1, I14T5, I15T1 and I16T1(MobileClient) \\ \hline
			\end{tabular}
			\vspace{1cm}
			\begin{tabular}{p{4cm} | p{8cm}} \hline
				\textbf{Test Procedure ID} & TP4 \\ \hline
				\textbf{Purpose} & This test procedure verifies whether a Guest can make a request:
				\begin{itemize}
					\item can fill the request form
					\item can create and store the request
					\item can submit the request to a TaxiDriver
				\end{itemize}
				\\ \hline
				\textbf{Procedure Steps} & I1T1, I3T2, I5T2, I6T1, I7T1, I12T3 \\ \hline
			\end{tabular}
			\vspace{1cm}
			\begin{tabular}{p{4cm} | p{8cm}} \hline
				\textbf{Test Procedure ID} & TP5 \\ \hline
				\textbf{Purpose} & This test procedure verifies whether a RegisteredPassenger can make a request:
				\begin{itemize}
					\item can create and store the request
					\item can submit the request to a TaxiDriver
				\end{itemize}
				\\ \hline
				\textbf{Procedure Steps} &  \\ I1T1, I4T2, I5T2, I6T1, I8T1, I13T1 \\ \hline
			\end{tabular}
			\vspace{1cm}
			\begin{tabular}{p{4cm} | p{8cm}} \hline
				\textbf{Test Procedure ID} & TP6 \\ \hline
				\textbf{Purpose} & This test procedure verifies whether a TaxiDriver can answer to a request:
				\begin{itemize}
					\item can view the notification of a request
					\item can answer positively
					\item can answer negatively
				\end{itemize}
				\\ \hline
				\textbf{Procedure Steps} &  \\ I5T2, I7T1, I8T1, I10T1(positive). I10T2(negative) \\ \hline
			\end{tabular}
			\vspace{1cm}
			\begin{tabular}{p{4cm} | p{8cm}} \hline
				\textbf{Test Procedure ID} & TP7 \\ \hline
				\textbf{Purpose} & This test procedure verifies whether a RegisteredPassenger can make a reservation:
				\begin{itemize}
					\item can fill the reservation form
					\item can create and store the reservation
				\end{itemize}
				\\ \hline
				\textbf{Procedure Steps} & I2T1, I8T1, I9T1, I13T2 \\ \hline
			\end{tabular}
			\vspace{1cm}
			\begin{tabular}{p{4cm} | p{8cm}} \hline
				\textbf{Test Procedure ID} & TP8 \\ \hline
				\textbf{Purpose} & This test procedure verifies whether a RegisteredPassenger can delete a reservation:
				\begin{itemize}
					\item can view his reservations
					\item can delete the selected reservation
				\end{itemize}
				\\ \hline
				\textbf{Procedure Steps} & I2T3, I2T2, I13T3\\ \hline
			\end{tabular}
