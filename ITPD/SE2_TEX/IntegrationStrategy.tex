\section{Integration Strategy}
	\subsection{Entry Criteria}
	To proceed at the Integration Testing level treated in this document is necessary that every
	component of the system has been previously unit tested with a positive result.
	We must be sure that all the methods inside the components, here considered like black boxes,
	work as expected.
	This is a necessary constraint that must be satisfied, because the purpose of the tests discussed in this
	document is to check the interaction between components (that's why it is called "integration testing").
	\subsection{Elements to be Integrated}
	Our system has been originally divided in four subsystems:
	\begin{itemize}
		\item Client
		\item Frontend System
		\item Backend System
		\item Database
	\end{itemize}
	Every subsystem is composed by a variable number of components and every single component has been
	unit tested individually. In this document we want to test the interaction between different components,
	so we are going to test all the interfaces provided by the components, dedicating a test case to every
	different interaction that involves an interface. Practically we want to test all the couples of components
	that interact between them.
	\paragraph{Integration Test of the UserClient - software}
	\begin{tabular}{p{2cm} | p{10cm}} \hline
		\textbf{ID} & \textbf{Integration Test} \\ \hline
		I17 & MobileClient $\leftrightarrow$ System \\ \hline
		I18 & WebClient $\leftrightarrow$ System \\ \hline
	\end{tabular}
	\paragraph{Integration Test of the FrontEnd system - software}
	\begin{tabular}{p{2cm} | p{10cm}} \hline
		\textbf{ID} & \textbf{Integration Test} \\ \hline
		I12 & Guest $\rightarrow$ Request \\ \hline
		I13 & RegisteredPassenger $\rightarrow$ Request \\ \hline
		I14 & RegisteredPassenger $\rightarrow$ Reservation \\ \hline
		I15 & TaxiDriver $\rightarrow$ Request \\ \hline
		I16 & TaxiDriver $\rightarrow$ TaxiDriver \\ \hline
	\end{tabular}
	\paragraph{Integration Test of the Backend system - software}
	\begin{tabular}{p{2cm} | p{10cm}} \hline
		\textbf{ID} & \textbf{Integration Test} \\ \hline
		I6 & Request $\rightarrow$ Area \\ \hline
		I7 & Guest $\rightarrow$ Request \\ \hline
		I8 & RegisteredPassenger $\rightarrow$ Request \\ \hline
		I9 & RegisteredPassenger $\rightarrow$ Reservation \\ \hline
		I10 & TaxiDriver $\rightarrow$ Request \\ \hline
		I11 & TaxiDriver $\rightarrow$ QueueManager \\ \hline
	\end{tabular}
	\paragraph{Integration Test of the Database - software}
	\begin{tabular}{p{2cm} | p{10cm}} \hline
		\textbf{ID} & \textbf{Integration Test} \\ \hline
		I1 & Request $\rightarrow$ Database \\ \hline
		I2 & Reservation $\rightarrow$ Database \\ \hline
		I3 & Guest $\rightarrow$ Database \\ \hline
		I4 & RegisteredPassenger $\rightarrow$ Database \\ \hline
		I5 & TaxiDriver $\rightarrow$ Database \\ \hline
	\end{tabular}
	\subsection{Integration Testing Strategy}
	We have chosen a bottom-up strategy. We think it's the best choice because represents a more gradual
	approach to the system testing. It also allows to find an eventual issue at the deepest level of
	integration, giving the possibility to intervene immediately at the correct level, without having to
	resort to a tougher analysis.
	\subsection{Sequence of Component/Function Integration}
		\subsubsection{Software Integration Sequence}
		\subsubsection{Subsystem Integration Sequence}