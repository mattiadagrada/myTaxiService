\section{Functional Point Approach}
The Functional Point approach is a technique that is used to evaluate the effort
needed for the design and implementation of an application.
This technique evaluates an application analysing the functionalities of the application itself.
For what concerns our application all the functionalities can be derived from the RASD document and from
the Design Document. Each functionality is evaluated in its totality.
This technique groups the functionalities of an application in:
\begin{itemize}
 \item Internal Logic File: it represents a set of homogeneous data handled
by the system. In our application, MyTaxiDriver, in this category are included all the
data structure saved into the database.
 \item External Interface File: it represents a set of homogeneous data used
by the application but generated and handled by external application. In our application, MyTaxiDriver,
in this category is included the map service offered by GoogleMaps.
 \item External Input: elementary operation to elaborate data coming from the external environment.
 \item External Output: elementary operation that generates data for the external environment.
 \item External Inquiry: elementary operation that involves input and output.
operations.
\end{itemize}
The following table outline the number of Functional Point based on funtionality
and relative complexity:
\iffalse
\\ \\ \begin{tabular} { l | l | l  } \hline
	\multicolumn{2}{ | c | }{$Function Point$} & \multicolumn{3}{ | c | }{$Complexity$} \\ \hline


\end{tabular}
\fi
\subsection{Internal Logic File}
\begin{tabular} {| l | l | r |} \hline
  Internal Logic File & Complexity & FP \\ \hline
  User & Simple & 7 \\
  Request & Simple & 7 \\
  Reservation & Simple & 7 \\
  Area & Average & 10 \\ \hline
  Total & \multicolumn{2}{ | r |}{31} \\ \hline
\end{tabular}
\subsection{External Logic File}
The application has to manage the maps coming from the external interfaces provided by Google Maps
in order to calculate taxi paths and show the position of the users.
A map is a complex set of data that probably must be elaborated before to be used.
So it is necessary that the developer in charge of building this functionality studies
Google Maps APIs as well as possible, a task that could require an important amount of time.
\\ \\ \begin{tabular} {| l | l | r |} \hline
  External Logic File & Complexity & FP \\ \hline
  Map & Complex & 10 \\ \hline
  Total & \multicolumn{2}{ | r |}{10} \\ \hline
\end{tabular}
\subsection{External Input}
\begin{tabular} {| l | l | r |} \hline
  External Input & Complexity & FP \\ \hline
  Login/Logout & Simple & 3 \\ \hline 
  Registration & Simple & 3 \\
  Make Request & Complex & 6 \\
  Make/Delete Reservation & Complex & 6 \\
  Give/Revoke Availability & Average & 4 \\
  Accept/Refuse Request & Average & 4 \\ \hline
  Total & \multicolumn{2}{ | r |}{26} \\ \hline
\end{tabular}
\subsection{External Output}
The application does not create important objects for the output. The only external output
can be considered the generation of the notifications, which are simple messages shown on
the device screen.
\\ \\ \begin{tabular} {| l | l | r |} \hline
  External Output & Complexity & FP \\ \hline
  Notificaiton & Simple & 4 \\ \hline
  Total & \multicolumn{2}{ | r |}{4} \\ \hline
\end{tabular}
\subsection{External Inquiries}
Our application is mainly concentrated on the request functionality, so there are many
inputs, but only few inquiries. The only inquiries we have, are:
\begin{itemize}
	\item the list of reservation, simple objects, to let the registered passenger select
	the reservation to delete
	\item the taxi profile, another simple set of data, where he can set his availability.
\end{itemize}
\begin{tabular} {| l | l | r |} \hline
  External Inquiries & Complexity & FP \\ \hline
  Reservation & Simple & 3 \\
  Taxi driver profile & Simple & 3 \\ \hline
  Total & \multicolumn{2}{ | r |}{6} \\ \hline
\end{tabular}


Totale = 77

LOC = 46*77 = 3542

effort = $2.94*EAF*(3542)^1.0997$ = 11.81 Person/Months

Duration = $3.67 * (11.81)^0,3179$ = 8.04

Number of people = $effort / Duration = 11.81 / 8.04 = 1.47 \rightarrow 2 $

\section{Tasks}
	\subsection{Identification}
	We present a brief description of the tasks that our project requires to be executed.
	\begin{description}
		\item [T1: Requirements Analysis and Specification] \hfill \\
			This is the first step of the project, which implies a deep analysis of the requirements
			submitted by the customer and the drafting of a proper document. It is very important that
			the requirement analysis and transformation is very clear and close to the customer necessities.
		\item [T2: Design definition] \hfill \\
			In this step, we choice the architecture of our whole system that will be the base of the
			project development. As T1, this task requires the drafting of a proper document.
		\item [T3: Test Plan] \hfill \\
			In this step we have to think about the plan to test our system. We define what has to be tested,
			when and with which other components. This step too requires a proper document.
		\item [T4: Project Plan] \hfill \\
			We define how much time we will have to spend on the project, the risks that we could face during
			and after the whole development process and allocate our available time to perform the defined tasks.
		\item [T5: Development of system] \hfill \\
			The coding part of the project, we have to write the real application based on the previous documents.
			It can be divided in the development of the various subsystems:
			\begin{itemize}
				\item Database
				\item Backend system
				\item Frontend system
				\item Clients
			\end{itemize}
		\item [T6: Unit testing] \hfill \\
			In this task we have to test each component independently from the others
		\item [T7: Integration testing] \hfill \\
			In this step we have to start integrating the various components each other in order to test the
			different functionalities across the single entities and subsystems. 
		\item [T8: System testing] \hfill \\
			This is the final test of the system. Global functionalities are tested and some performance tests
			are performed to improve the system usability.
	\end{description}
	\subsection{Allocation}