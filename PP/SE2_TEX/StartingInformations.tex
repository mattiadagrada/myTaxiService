\section{Functional Point Approach}
The Functional Point approach is a technique that is used to evaluate the effort
needed for the design and implementation of an application.
This technique evaluates an application analysing the functionalities of the application itself.
For what concerns our application all the functionalities can be derived from the RASD document and from
the Design Document. Each functionality is evaluated in its totality.
This technique groups the functionalities of an application in:
\begin{itemize}
  \item Internal Logic File: it represents a set of homogeneous data handled
by the system. In our application, MyTaxiDriver, in this category are included all the
data structure saved into the database.
 \item External Interface File: it represents a set of homogeneous data used
by the application but generated and handled by external application. In our application, MyTaxiDriver,
in this category is included the map service offered by GoogleMaps.
 \item External Input: elementary operation to elaborate data coming from the external environment.
 \item External Output: elementary operation that generates data for the external environment.
 \item External Inquiry: elementary operation that involves input and output.
operations.
\end{itemize}
The following table outline the number of Functional Point based on functionality
and relative complexity:
\begin{center}
\begin{tabular} { | l | c | c | c | } \hline
  \multirow{2}{*}{Function Point} & \multicolumn{3}{|c|}{Complexity} \\ \cline{2-4}  
  & Simple & Average & Complex \\ \hline 
  Internal Logic File & 7 & 10 & 15 \\ \hline
  External Interface File & 5 & 7 & 10 \\ \hline
  External Input & 3 & 4 & 6 \\ \hline
  External Output & 4 & 5 & 7 \\ \hline
  External Inquiry & 3 & 4 & 6 \\ \hline
\end{tabular}
\end{center}

\subsection{Internal Logic File}
The system stores only little information about users. Guests are
stored only if they make a request and only by adding their IP address, while
for registered users and taxi drivers are stored personal data.
Request and reservations are stored in the system in order to have an history
always accessible. Requests are stored saving the position of the request and the
IP of the user, while reservations are stored saving the origin and the destination
and the informations of the user who made it.
Areas are considered of high complexity due to the fact that they are stored along
with the associated queue.\\
\begin{center}
\begin{tabular} { | l | l | r | } \hline
  Internal Logic File & Complexity & FP \\ \hline
  User & Simple & 7 \\ \hline
  Request & Simple & 7 \\ \hline
  Reservation & Simple & 7 \\ \hline
  Area & Complex & 15 \\ \hline
  \multicolumn{2}{ | l |}{Total} & 36 \\ \hline
\end{tabular}
\end{center}

\subsection{External Logic File}
The application has to manage the maps coming from the external interfaces provided by Google Maps
in order to calculate taxi paths and show the position of the users.
A map is a complex set of data that probably must be elaborated before to be used.
So it is necessary that the developer in charge of building this functionality studies
Google Maps APIs as well as possible, a task that could require an important amount of time.
\begin{center}
\begin{tabular} { | l | l | r |  } \hline
  External Interface File & Complexity & FP \\ \hline
  Map & Complex & 10 \\ \hline
  \multicolumn{2}{ | l |}{Total} & 10 \\ \hline
\end{tabular}
\end{center}

\subsection{External Input}
This section involves all the possible way of interaction between the user and
the application.
\begin{itemize}
  \item Login/Logout: these are simple operations related only to the client and
  the user on the frontend server.
  \item Registration: this is a simple operation which again, as the login/logout is
  related only to the client and the user on the frontend server, with the addition of the
  storage of the newly created RegisteredPassenger into the database.
  \item Make Request: this operation is considered complex because not only it requires
  to insert personal data, to retrieve the GPS position and to store the IP, but also starts
  the process of searching an available taxi to whom submit the request, therefore involving Area,
  Queue and TaxiDriver.
  \item Make Reservation: this operation is considered complex since it requires to store some information
  into the database but also once the timer expires it will start a request process, which is considered
  complex as said above.
  \item Delete Reservation: this operation is considered average because it requires the retrieval of
  the list of reservations made by a RegisteredPassenger from the database at first, and then delete itself.
  \item Give/Revoke Availability: these operations are considered average since they involve a TaxiDriver and the
  QueueManager which will add or remove the TaxiDriver from the Queue he currently is in.
  \item Accept/Refuse Request: these operations are considered average since they involve a TaxiDriver and a Request
  but also the QueueManager because the TaxiDriver is preventively removed from the Queue when he receive a Request notification.
  In case the TaxiDriver refuses the request, the QueueManager has to place him again in Queue.
\end{itemize}
\begin{center}
\begin{tabular} { | l | l | l |  } \hline
  External Input & Complexity & FP \\ \hline
  Login/Logout & Simple & $2 \times 3$ \\ \hline
  Registration & Simple & 3 \\ \hline
  Make Request & Complex & 6 \\ \hline
  Make Reservation & Complex & 6 \\ \hline
  Delete Reservation & Average & 4 \\ \hline
  Give/Revoke Availability & Average & $2 \times 4$ \\ \hline
  Accept/Refuse Request & Average & $2 \times 4$ \\ \hline
  \multicolumn{2}{ | l |}{Total} & 43 \\ \hline
\end{tabular}
\end{center}

\subsection{External Output}
The application does not create important objects for the output. The only external output
can be considered the generation of the notifications, which are simple messages shown on
the device screen.
\begin{center}
\begin{tabular} { | l | l | r | } \hline
  External Interface File & Complexity & FP \\ \hline
  Notificaiton & Simple & 4 \\ \hline
  \multicolumn{2}{ | l |}{Total} & 4 \\ \hline
\end{tabular}
\end{center}
\subsection{External Inquiries}
Our application is mainly concentrated on the request functionality, so there are many
inputs, but only few inquiries. The only inquiries we have, are:
\begin{itemize}
	\item the list of reservation, simple objects, to let the registered passenger select
	the reservation to delete
	\item the taxi profile, another simple set of data, where he can set his availability.
\end{itemize}
\begin{center}
\begin{tabular} {|  l | l | r | } \hline
  External Interface File & Complexity & FP \\ \hline
  Reservation & Simple & 3 \\ \hline
  Taxi driver profile & Simple & 3 \\ \hline
  \multicolumn{2}{ | l |}{Total} & 6 \\ \hline
\end{tabular}
\end{center}
\begin{center}
\begin{itemize}
  \item Total number of Function Points: we have computed the following value of
  FP: 99. We will now estimate the size of the application in KLOC and use COCOMO
  in order to estimate the effort.
\end{itemize}
\end{center}

\section{COCOMO}

To estimate the SLOC starting from the FP we use an average conversion factor of 46
as described at http://www.qsm.com/resources/function-point-languages-table, an updated
version that adds J2EE of the table included in official manual
(http://sunset.usc.edu/research/COCOMOII/Docs/modelman.pdf).

\begin{center}
      $ LOC = 46 \times 99 = 4554 $
\end{center}

We will consider our application as a project with all "nominal" Cost Drivers and
Scale Drivers. Therefore we will use an EAF equal to $1.00$ and exponent E equal to $1.0997$
We Consider a project with all “Nominal” Cost Drivers and Scale Drivers would have an EAF
of 1.00 and exponent E of 1.0997.
With these values we can proceed and calculate the effort estimation:
\begin{center}
		$  effort = 2.94 \times EAF \times (KSLOC)^E$ \\
		$  effort = 2.94 \times EAF \times (4.554)^1.0997= 15.57 \  ^{Person}/_{Month}$
\end{center}
Where
\begin{itemize}
	\item EAF: Effort Adjustment Factor which is derived from Cost Drivers.
	\item E: Exponent derived from Scale Drivers.
\end{itemize}

We can then calculate the expected duration of the project in month
\begin{center}
	$ Duration = 3.67 \times (effort)^E $ \\
	$ Duration = 3.67 \times (15.57)^{0.3179} = 8.78 months $
\end{center}

And finally we can estimate the number of people needed to complete the project

\begin{center}
	$ N_{people} = \frac {effort}{Duration}$ \\
	$ N_{people} = \frac {15.57}{8.78} = 1.77 \rightarrow 2 people $
\end{center}