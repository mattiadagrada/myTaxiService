\section{Tasks}
\subsection{Identification}
We present a brief description of the tasks that our project requires to be executed.
\begin{description}
	\item [T1: Requirements Analysis and Specification] \hfill \\
	This is the first step of the project, which implies a deep analysis of the requirements
	submitted by the customer and the drafting of a proper document. It is very important that
	the requirement analysis and transformation is very clear and close to the customer necessities.
	\item [T2: Design definition] \hfill \\
	In this step, we choice the architecture of our whole system that will be the base of the
	project development. As T1, this task requires the drafting of a proper document.
	\item [T3: Test Plan] \hfill \\
	In this step we have to think about the plan to test our system. We define what has to be tested,
	when and with which other components. This step too requires a proper document.
	\item [T4: Project Plan] \hfill \\
	We define how much time we will have to spend on the project, the risks that we could face during
	and after the whole development process and allocate our available time to perform the defined tasks.
	\item [T5: Development of system] \hfill \\
	The coding part of the project, we have to write the real application based on the previous documents.
	It can be divided in the development of the various subsystems:
	\begin{itemize}
		\item Database
		\item Backend system
		\item Frontend system
		\item Clients
	\end{itemize}
	\item [T6: Unit testing] \hfill \\
	In this task we have to test each component independently from the others
	\item [T7: Integration testing] \hfill \\
	In this step we have to start integrating the various components each other in order to test the
	different functionalities across the single entities and subsystems.
	\item [T8: System testing] \hfill \\
	This is the final test of the system. Global functionalities are tested and some performance tests
	are performed to improve the system usability.
\end{description}
There are some dependencies that must be respected and that limit the number of parallel tasks that can be executed.
Generally, given the order of the tasks above, we can say that every task requires that all the previous tasks 
must be consolidated, before to be considered. We will see particular cases in the allocation.
\subsection{Allocation}
As premise we have to consider that we are still students, so our knowledge, experience and time
to dedicate at this project are quite limited.
\\ \\
\begin{tabular}{|p{2cm}|p{5cm}|p{5cm}|} \hline
	Step & Cesare Bernardis & Mattia Dagrada \\ \hline
	1 & T1 & T1 \\ \hline
	2 & T2 & T2 \\ \hline
	3 & T3 & T3 \\ \hline
	4 & T4 & T4 \\ \hline
	5 & T5, T6 & T5, T6 \\ \hline
	6 & T7 & T7 \\ \hline
	7 & T8 & T8 \\ \hline
\end{tabular}
\\ \\ \\
The first thing to consider is that our experience is very limited, so in most of the cases two
thinking heads are better than only one. This is why we decided to analyse the project and write the
relative documents together, in order to share our individual knowledge and to be both aware of the
project construction. The individualities come out during the development process, where the system can
be split up in different parts to save time. The last tests require both the parts of the project, so
we will be both involved. Let's analyse the steps a bit deeper:
\begin{description}
	\item[Steps 1,2,3,4] \hfill \\
	These steps are preliminary analyses and documents, so, as said before, we think could be a good choice
	to spend some time thinking together about the requirements, design, architecture and development process
	of our project. Maybe the various parts of each document can be divided between us, but we are both focused
	on the same document (and task).
	\item[Step 5] \hfill \\
	The development process can be divided in two branches, trying to keep the two parts with the same size.
	So we can think that one of us writes the code of the Clients and the Frontend system, while the other writes
	the code of the backend and the database interface. Subsequently every component of the team can "unit test"
	the modules that he wrote. If one of the developers ends before the others can help with the development of
	the other parts or maybe prepare the tests for the Unit testing of those parts.
	\item[Steps 6,7] \hfill \\
	The project ends with the integration and test of the complete functionalities of the system. To accomplish
	these tasks the code must be complete, so all the team must have completed the implementation of each part.
	Eventual errors may be found and corrected by the same person that wrote the wrong code.
\end{description}