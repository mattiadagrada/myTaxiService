\section{Risk and Recoveries}
In the development of a project, handling the risks is a very important task. The first thing to do is to identify all
the most relevant risks that can occur. Then it is important to determine recovery actions in order to solve the risks.
In our project the main risks that can be found are:
\begin{itemize}
  \item \textbf{Schedule.} It may happen that there are difficulties following the project schedule, which means that the project will be finished later
  than the estimated time. This can lead to an increase in effort and cost.
  In order to avoid this kind of problem, that is one of the main and more frequent problem that happens in doing a project, it is important to monitor all the phases of the project,
  in order to understand if something is slowing down and why.
  \item \textbf{Technology.} Another risk that can be critical. At the moment of deploying the project on an infrastructure the requirements may be higher than expected
  and lead to an increase in the overall cost of the project.
  A good strategy can be to overestimate the requirements in the beginning.
  \item \textbf{Lack of experience.} If the team developing the project doesn't have a good amount of experience there can be problems while developing some parts of the project since
  they may result too difficult for the developer and this may lower the overall quality of the project.
  Therefore it is important to specify at best all the requirements for the project, in this way the team of developers can be properly selected without facing this kind of problem.
  \item \textbf{Budget.} It may happen that the initial budget assigned to the project may not be enough to cover the complete development of the project and if there are problems in recovering
  the needed money the entire project may be slowed down or stopped.
  It is also difficult to prevent this kinda of problem because if the starting budget is limited it can be hard to find more money. A good strategy can eventually be to find more investors and to
  wait before starting the project untill the starting budget seems fine.
\end{itemize}

We will adopt a proactive strategy, that tries to avoid risk in advance instead of trying to solve them after they are already become a problem. All the potential threat will be
identified and analyzed, also taking in consideration the probability of happening and the potential damage. This analysis produces a ranking of risks, allowing to create a contingency plan
that will give priority to the risks higher in the ranking. 
