\section{Algorithm Design}
	\subsection{Request management}
	This algorithm
	\subsubsection{Pseudocode}
	\begin{algorithm}
		\begin{algorithmic}[1]
			\Procedure {MakeRequest}{user}
			\If {$!validRequest(user)$}
			\State $sendMessage(SPAM\_REQUEST\_ERROR, user)$
			\Else
			\State {$rqst \gets new\  Request(user)$}
			\State {$storeRequest(rqst)$}
			\State {$area \gets retrieveArea(origin)$}
			\State {$taxi \gets area.queue.remove(0)$}
			\State {$taxi.submitRequest(rqst)$}
			\EndIf
			\EndProcedure
		\end{algorithmic}
	\end{algorithm}
	\begin{algorithm}
		\begin{algorithmic}[1]
			\Procedure {SubmitRequest}{request}
			\State {$timer \gets new Timer(60)$}
			\State {$timer.start()$}
			\State {$sendMessage(rqst, taxi)$}
			\EndProcedure
		\end{algorithmic}
	\end{algorithm}
	\begin{algorithm}
		\begin{algorithmic}[1]
			\Procedure {RequestResponse} {answer, request}
			\State{$request.manageAnswer(answer)$}
			\If {$answer \equiv REFUSE$}
			\State{$ taxi.declined++ $}
			\If{$ taxi.declined \equiv 3 $}
			\State {$ taxi.setUnavailable() $}
			\Else
			\State{$ queueManager.add(taxi) $}
			\EndIf
			\Else
			\State {$ taxi.setUnavailable() $}
			\EndIf
			\EndProcedure
		\end{algorithmic}
	\end{algorithm}
	\begin{algorithm}
		\begin{algorithmic}[1]
			\Procedure {ManageAnswer}{answer}
			\If{$ answer \equiv REFUSE $}
			\State{$area \gets retrieveArea(rqst.origin)$}
			\State{$taxi \gets area.queue.remove(0)$}
			\State{$taxi.submitRequest(rqst)$}
			\Else
			\State {$user.sendConfirmation(estimatedTime(), taxi)$}
			\EndIf
			\EndProcedure
		\end{algorithmic}
	\end{algorithm}
	\newpage
	\subsection{Queue balancing}
	This algorithm shows the mechanism that every two minutes balances the area queues.
	The idea is that a taxi may move from its position at any time, for any reason, and, perhaps,
	pass through other areas than the one he belonged to when he was first inserted into the system.
	So a simple job running this algorithm can keep the system up to date and avoid inconsistencies
	(like sending a notification to a taxi out of the interested area).
		\subsubsection{Pseudocode}
		\begin{algorithm}
			\begin{algorithmic}[1]
				\ForAll {$TaxiDrivers\textnormal{ available as }TaxiDriver$}
					\State $actualArea \gets TaxiDriver.Area$
					\State $oldArea \gets TaxiDriver.Queue.Area$
					\If{$actualArea \neq oldArea$}
						\State $oldArea.\textnormal{extract}(TaxiDriver)$
						\State $actualArea.\textnormal{insert}(TaxiDriver)$
					\EndIf
				\EndFor
				\State $criticalAreas \gets \textnormal{List}$
				\ForAll{$Areas \textnormal{ as } Area$}
					\If{$Area.Queue.size \leq Area.minCriticalSize$} 
						\State $criticalAreas.\textnormal{add}(Area)$
					\EndIf
				\EndFor
				\State $surplusTaxis \gets \textnormal{List}$
				\ForAll{$Queues \textnormal{ as } Queue$}
					\ForAll{$TaxiDrivers \textnormal{ with index } \geq Queue.Area.maxCriticalSize$}
						\State $surplusTaxis.\textnormal{add}(TaxiDriver)$
					\EndFor
				\EndFor
				\State $nofitication \gets$ "In the areas ".$criticalAreas$.toString()." we have a taxi shortage.
					Go there and get back to work!"
				\ForAll {$surplusTaxis \textnormal{ as } TaxiDriver$}
					\State $TaxiDriver.\textnormal{sendNotification}(notification)$
				\EndFor
			\end{algorithmic}
		\end{algorithm}
		\subsubsection{Clarifications}
		The first section of the algorithm updates the area queues, moving taxis from the areas they belonged last
		time they were inserted to the one they belong now, only if they are different (we assume that
		\textit{TaxiDriver.Area} retrieves the actual position of the taxi and returns the corresponding area).
		After the update, the system tries to balance his queues where the values are out of some critical values.
		It is simple: if the length of an area queue is under a \textit{minCricitalSize}, that area needs to be
		"filled" with taxis. If the length of an area queue is above a \textit{maxCricitalSize}, that area has
		too many taxis without a reason an the last taxis in those queues can be moved elsewhere.
		So the system notifies these \textit{surplusTaxis}, advising them to move where they are really needed.